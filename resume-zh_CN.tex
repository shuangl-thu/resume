% !TEX TS-program = xelatex
% !TEX encoding = UTF-8 Unicode
% !Mode:: "TeX:UTF-8"

\documentclass{resume}
\usepackage{graphicx}
\usepackage{tabu}
\usepackage{multirow}
\usepackage{zh_CN-Adobefonts_external} % Simplified Chinese Support using external fonts (./fonts/zh_CN-Adobe/)
%\usepackage{zh_CN-Adobefonts_internal} % Simplified Chinese Support using system fonts
\usepackage{linespacing_fix} % disable extra space before next section
\usepackage{cite}

\begin{document}
\pagenumbering{gobble} % suppress displaying page number

\name{李爽}

\basicInfo{
  \email{seulishuang@163.com} \textperiodcentered\ 
  \phone{(+86) 188-0019-9283} \textperiodcentered 
  \linkedin[Shuang Li]{https://www.linkedin.com/in/shuang-li-454a3185/}
}
 
\section{\faGraduationCap\  教育背景}
\datedsubsection{\textbf{清华大学} \quad \textbf{博士} \quad 计算机科学与技术 \hfill 推荐算法/机器学习/强化学习}{2013.8 -- 预计2019.1}

\datedsubsection{\textbf{东南大学} \quad \textbf{学士} \quad 软件工程 \quad 绩点排名: 4/135}{2008.8 -- 2013.6}

\section{\faUsers\ 科研经历}
\datedsubsection{\textbf{基于强化学习的方法进行多样化学习排序}}{2018.1 -- 2018.5}
\textit{\textbf{\qquad Recommendation Diversification,LSTM,RL,Actor-Critic,Pytorch}}
\begin{onehalfspacing}
	\begin{itemize}
		\item 提出一种新的Actor-Critic强化学习算法来进行推荐列表的多样化学习,样本效率和稳定性得到显著提升.
		\item 使用LSTM模型来建模用户状态的变化,用以考虑同一个推荐列表中不同物品的相互关系.
		\item 基于MovieLens和Pytorch进行实验设计,与不同的监督学习和强化学习算法进行对比.
	\end{itemize}
\end{onehalfspacing}
\datedsubsection{\textbf{基于物品的图片和文本描述信息提高评分预测准确度}}{2017.3 -- 2018.1}
\textit{\textbf{\qquad Matrix Factorization,DNN,CNN,Attention,Keras}}
\begin{onehalfspacing}
\begin{itemize}
	\item 结合传统的矩阵分解方法和最新提出的深度学习算法,同时融入文本和图像信息来提高评分预估的准确性.
	\item 采用Attention机制自适应加权不同离散文本标签的Embedding向量,采用CNN提取图片的连续压缩特征.
	\item 基于Movielen数据集和Keras平台完成不同分解算法的对比试验.
\end{itemize}
\end{onehalfspacing}
\datedsubsection{\textbf{Pair-Wise的多样化学习排序研究}}{2016.12 -- 2017.3}
\textit{\textbf{\qquad Learning to Rank,Matrix Factorization,Pair-Wise,Python}}
\begin{onehalfspacing}
	\begin{itemize}
		\item 采用Pair-Wise的Learning to Rank方法学习多样化推荐排序.
		\item 直接对评分矩阵进行分解,End-to-End地优化最终的多样化排序列表.
	\end{itemize}
\end{onehalfspacing}
\section{\faUsers\ 项目经历}
\datedsubsection{基于室内定位和HTTP数据的挖掘和应用 \quad python, regex, matplotlib}{2015.8 -- 2016.3}
%\begin{onehalfspacing}
%\begin{itemize}
%  \item 通过HTTP数据挖掘用户搜索偏好与用户活动背景,设备类型等因素的关联
%  \item 通过室内定位数据检测办公室活动(会议开始时间, 结束时间, 参与人数)
%  \item 通过MAC地址识别用户设备类型
%  \item 使用技术: python, 网页抓取, 正则表达式, 可视化
%\end{itemize}
%\end{onehalfspacing}

\datedsubsection{博士生实践: 甘孜州手机旅游APP \qquad\qquad Objective C}{2015.6 -- 2015.8}
%\begin{onehalfspacing}
%为游客及管理部门创造一个信息交互平台,提供旅游攻略分享、旅游资讯发布、门票酒店预订等功能
%\begin{itemize}
%	\item 负责ios端功能实现
%	\item 使用技术: Objective C, 数据库
%\end{itemize}
%\end{onehalfspacing}
\datedsubsection{软件创新大赛:智能交通服务平台 \qquad\qquad J2ME, google maps api}{2010.9 -- 2010.12}
%\role{\LaTeX, Python}{个人项目}
%\begin{onehalfspacing}
%以物联网技术方式实现车位信息导航、智能的士导航、路况提示等示范功能
%\begin{itemize}
%  \item 负责乘客手机终端  通过GPS模块和Google地图API采集用户位置信息并与中心服务器交互
%  \item 编程语言: Java
%\end{itemize}
%\end{onehalfspacing}

\section{\faUsers\ 实习经历}
\datedsubsection{\textbf{IBM} (上海) \qquad 可视化分析ZoS性能 \qquad Eclipse, Java, SWT, XML}{2012.6 -- 2012.12}
%\begin{itemize}
%	\item 扩展用于可视化分析大型主机Zos系统性能的工具VPA的功能
%	\item 涉及SWT、XSD、HSQL等技术, 完成设计、编码、测试等工作及相应的文档编写
%\end{itemize}

\section{\faBook\ 论文发表}
\begin{itemize}
	\item \textbf{S. Li}, Y. Y, Y. Zhou, Y. Zhang. "Sample-Efficient Actor-Critic Reinforcement Learning for Recommendation Diversification." 在投
	\item \textbf{S. Li}, Y. Y, C. Wu, K. Zhao, Y. Zhou, Y. Zhang. "Deep Collaborative Filtering Incorporating Auxiliary Multi-Media Information." In 16th IEEE UIC, Guangzhou, 2018.
	\item \textbf{S. Li}, X. Lan, Y. Zhou, Y. Zhang. "Exploring and Understanding Web Search Behavior with Human Activities." In 15th IEEE UIC, San Francisco, 2017.
	\item \textbf{S. Li}, Y. Zhou, D. Zhang, Y. Zhang, X. Lan. "Learning to Diversify Recommendations Based on Matrix Factorization." In IEEE CyberSciTech, Orlando, 2017, pp. 68-74.
\end{itemize}

\section{\faTrophy\ 荣誉奖项}
\begin{itemize}[parsep=0.5ex]
	\item 清华之友-搜狐研发奖学金(2017),清华大学综合优秀奖学金(2016) 
	%	\item 清华大学综合优秀奖学金(2016) 
	\item 东南大学国家奖学金(2012),东南大学金蝶奖学金(2011)
	%	\item 东南大学金蝶奖学金(2011)
	\item 国际数模竞赛 二等奖(2012),国家数模竞赛 江苏省二等奖(2011),大学生软件创新大赛一等奖(2010)
	%	\item 国家大学生数学建模竞赛 江苏省二等奖(2011)
	%\item 第三届全国大学生软件创新大赛一等奖(2010)
\end{itemize}

%\section{\faCogs\ 专业技能}
%\begin{itemize}[parsep=0.5ex]
%	\item 编程语言: Python > C++ > Java > C
%	\item 熟练Linux、Shell、Vim、Git、Pandas、Matplotlib
%	\item 了解Hadoop、MapReduce
%	\item 深度学习框架: 熟悉Keras、Pytorch,Numpy
%	\item 熟悉深度学习,强化学习等领域算法
%\end{itemize}
\section{\faCogs\ 专业技能}
熟悉python>C++>Java, Keras, Pytorch, Linux, Shell, Vim, Git, Pandas, Matplotlib; 了解Hadoop, MapReduce
\end{document}

