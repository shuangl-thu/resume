% !TEX TS-program = xelatex
% !TEX encoding = UTF-8 Unicode
% !Mode:: "TeX:UTF-8"

\documentclass{resume}
\usepackage{graphicx}
\usepackage{tabu}
\usepackage{multirow}
\usepackage{zh_CN-Adobefonts_external} % Simplified Chinese Support using external fonts (./fonts/zh_CN-Adobe/)
%\usepackage{zh_CN-Adobefonts_internal} % Simplified Chinese Support using system fonts
\usepackage{linespacing_fix} % disable extra space before next section
\usepackage{cite}

\begin{document}
\pagenumbering{gobble} % suppress displaying page number

\name{李爽}

\basicInfo{
  \email{seulishuang@163.com} \textperiodcentered\ 
  \phone{(+86) 188-0019-9283} \textperiodcentered 
%  \linkedin[billryan8]{https://www.linkedin.com/in/billryan8}
}
 
\section{\faGraduationCap\  教育背景}
\datedsubsection{\textbf{清华大学} \ \textbf{在读博士研究生} \ 计算机科学与技术}{2013年8月 -- 至今}
%\textit{在读博士研究生}\ 计算机科学与技术, 预计 2018 年 6 月毕业\\
\textit{研究方向:}\ 推荐算法,强化学习,机器学习
\datedsubsection{\textbf{东南大学} \ \textbf{学士} \ 软件工程~绩点排名: 4/135}{2008年8月 -- 2013年6月}
%\textit{学士}\ 软件工程  \绩点排名:4/135

\section{\faUsers\ 科研经历}
\datedsubsection{\textbf{基于强化学习的方法进行多样化学习排序}}{2018年1月 -- 至今 }
\textbf{~~~~~~~Recommendation Diversification,LSTM,RL,Actor-Critic,Pytorch}
\begin{onehalfspacing}
	\begin{itemize}
		\item 提出一种新的Actor-Critic强化学习算法来进行推荐列表的多样化学习,样本效率和稳定性得到显著提升.
		\item 使用LSTM模型来建模用户状态的变化,用以考虑同一个推荐列表中不同物品的相互关系.
		\item 基于MovieLens和Pytorch进行实验设计,与不同的监督学习和强化学习算法进行对比.
	\end{itemize}
\end{onehalfspacing}
\datedsubsection{\textbf{基于物品的图片和文本描述信息提高评分预测准确度}}{2017年3月 -- 2018年1月}
\textbf{~~~~~~~Matrix Factorization,DNN,CNN,Attention,Keras}
\begin{onehalfspacing}
\begin{itemize}
	\item 结合传统的矩阵分解方法和最新提出的深度学习算法,同时融入文本和图像信息来提高评分预估的准确性.
	\item 采用Attention机制自适应加权不同离散文本标签的Embedding向量,采用CNN提取图片的连续压缩特征.
	\item 基于Movielen数据集和Keras平台完成不同分解算法的对比试验.
\end{itemize}
\end{onehalfspacing}
\datedsubsection{\textbf{List-Wise的多样化学习排序研究}}{2016年12月 -- 2017年3月}
\textbf{~~~~~~~Learning to Rank,Matrix Factorization,List-Wise,Python}
\begin{onehalfspacing}
	\begin{itemize}
		\item 采用List-Wise的Learning to Rank方法学习多样化推荐排序算.
		\item 直接对评分矩阵进行分解,End-to-End的优化最终的多样化排序列表.
	\end{itemize}
\end{onehalfspacing}

\section{\faUsers\ 项目经历}
\datedsubsection{\textbf{清华大学-思科联合项目: 基于位置和HTTP数据的挖掘和应用}}{2015年8月 -- 2016年3月}
%\role{Golang, Linux}{个人项目,和富帅糕合作开发}
\begin{onehalfspacing}
%xxxx
\begin{itemize}
  \item 通过HTTP数据挖掘用户搜索偏好与用户活动背景,设备类型等因素的关联
  \item 通过室内定位数据检测办公室活动(会议开始时间, 结束时间, 参与人数)
  \item 通过MAC地址识别用户设备类型
%  \item 使用技术: python, 网页抓取, 正则表达式, 可视化
\end{itemize}
\end{onehalfspacing}

\datedsubsection{\textbf{博士生实践: 甘孜州手机旅游APP公共服务平台}}{2015年6月 -- 2015年8月}
\begin{onehalfspacing}
为游客及管理部门创造一个信息交互平台,提供旅游攻略分享、旅游资讯发布、门票酒店预订等功能
\begin{itemize}
	\item 负责ios端功能实现
%	\item 使用技术: Objective C, 数据库
\end{itemize}
\end{onehalfspacing}
\datedsubsection{\textbf{软件创新大赛: 智能交通服务平台}}{2010 年9月 -- 2010 年12月}
%\role{\LaTeX, Python}{个人项目}
\begin{onehalfspacing}
以低成本、易部署的物联网技术方式实现车位信息导航、智能的士导航、路况提示等示范功能
\begin{itemize}
  \item 负责乘客手机终端  通过GPS模块和Google地图API采集用户位置信息并与中心服务器交互
%  \item 编程语言: Java
\end{itemize}
\end{onehalfspacing}


\section{\faUsers\ 实习经历}
\datedsubsection{\textbf{IBM} (上海) \ eclipse插件开发}{2012年6月 -- 2012年12月}
%\begin{itemize}
%	\item 扩展用于可视化分析大型主机Zos系统性能的工具VPA的功能
%	\item 涉及SWT、XSD、HSQL等技术, 完成设计、编码、测试等工作及相应的文档编写
%\end{itemize}

\section{\faCogs\ 专业技能}
% increase linespacing [parsep=0.5ex]
\begin{itemize}[parsep=0.5ex]
  \item 编程语言: Python > C++ > Java > C
  \item 熟练Linux、Shell、Vim、Git、Pandas、Matplotlib
  \item 了解Hadoop、MapReduce
  \item 深度学习框架: 熟悉Keras、Pytorch,Numpy
  \item 熟悉深度学习,强化学习等领域算法
\end{itemize}

\section{\faCogs\ 荣誉奖项}
% increase linespacing [parsep=0.5ex]
\begin{itemize}[parsep=0.5ex]
	\item 清华之友-搜狐研发奖学金(2017)
	\item 清华大学综合优秀奖学金(2016) 
	\item 东南大学国家奖学金(2012)
	\item 东南大学金蝶奖学金(2011)
	\item 国际大学生数学建模竞赛 二等奖(2012)
	\item 国家大学生数学建模竞赛 江苏省二等奖(2011)
	\item 第三届全国大学生软件创新大赛一等奖(2010)

\end{itemize}

\section{\faBook\ 论文发表}
\begin{itemize}
	\item Sample-Efficient Actor-Critic Reinforcement Learning for Recommendation Diversification. CIKM'18在投
	\item Deep Collaborative Filtering Incorporating Auxiliary Multi-Media Information, In 2018UIC
	\item Exploring and Undersanding Web Search Behavior with Human Activities, In 2017UIC
	\item Learning to Diversify Recommendations Based on Matrix Factorization, In 2017CyberSciTech
\end{itemize}
%\datedsubsection{Personalized Content Recommendation Based on Field Authorities in Transparent Computing}{2017IJAHUC}
%\datedsubsection{Exploring and Undersanding Web Search Behavior with Human Activities}{2017UIC}
%An xxx optimized for xxx\cite{verma2015large}
%\datedsubsection{Learning to Diversify Recommendations Based on Matrix Factorization}{2018MMM在投}

%% Reference
%\newpage
%\bibliographystyle{IEEETran}
%\bibliography{mycite}
\end{document}

